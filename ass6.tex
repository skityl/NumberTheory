\documentclass{unswmaths}

\usepackage{unswshortcuts}

\begin{document}

\subject{Number Theory}
\author{Edward McDonald}
\title{Assignment 6}
\studentno{3375335}


\setlength\parindent{0pt}

\newcommand{\Unit}{\mathbb{U}}
\newcommand{\modulo}[1]{\;\operatorname{mod}\;#1}
\newcommand{\pprime}{{p\text{ prime}}}

\unswtitle{}


\section*{Question 1}
\begin{lemma}
    For all $x > 0$, $\psi(x) \geq \vartheta(x)$.
\end{lemma}
\begin{proof}
    By definition, since
    \begin{equation*}
        \vartheta(x) = \sum_{\substack{\pprime,\\ p \leq x}} \log{p}
    \end{equation*}
    and
    \begin{equation*}
        \psi(x) = \sum_{n \leq x} \Lambda(n),
    \end{equation*}
    we have
    \begin{equation*}
        \psi(x) - \vartheta(x) = \sum_{\substack{\pprime,\\ p^n \leq x,\\ n \geq 2}} \log{p} \geq 0.
    \end{equation*}
    Hence, $\psi(x) \geq \vartheta(x)$.
\end{proof}
Hence, since $\psi(x) \leq 2x$, we conclude that $\vartheta(x) \leq 2x$.
\begin{lemma}
\label{logInequality}
    For $x > 0$, $\frac{\log{x}}{x^\alpha}$ has a maximum of $\frac{1}{e\alpha}$ where $\alpha > 0$. 
\end{lemma}
\begin{proof}
    For all real $s$,
    \begin{equation*}
        e^s \geq s+1.
    \end{equation*}
    Hence, $e^{s-1} \geq s$, and so
    \begin{equation*}
        e^s \geq es.
    \end{equation*}
    Substitute $s = \alpha\log{x}$, and hence
    \begin{equation*}
        x^\alpha \geq e\alpha\log{x}.
    \end{equation*}
    So we have the inequality,
    \begin{equation*}
        \frac{\log{x}}{x^\alpha} \leq \frac{1}{e\alpha}.
    \end{equation*}
    This inequality is sharp, since equality is attained for $x = e^\frac{1}{\alpha}$.    
%    By calculus, the function $f(x) = \frac{\log{x}}{x^\alpha}$ has a unique stationary point at $x = e^{1/\alpha}$. This point
%    must at least be a local maxima, since $f''(e^{1/\alpha}) = -\alpha e^{-(\alpha+2)/\alpha} < 0$. 
%    
%     Indeed, since $f(x) \rightarrow 0$ as $x \rightarrow \infty$, and $f(x) \rightarrow -\infty$ as $x\rightarrow 0$, this
%     must be a global maximum. 
%     
%     Hence, the global maximum of $f$ is $f(e^{1/\alpha}) = \frac{1}{e\alpha}$. 
\end{proof}

\begin{theorem}
\label{psiThetaBound}
    For $x > 0$,
    \begin{equation*}
        \psi(x)-\vartheta(x) \leq 9x^{\frac{1}{2}}
    \end{equation*}
\end{theorem}
\begin{proof}
    By definition,
    \begin{equation*}
        \psi(x)-\vartheta(x) = \sum_{\substack{m\geq 2,\\ p^m \leq x,\\ \pprime}} \log{p}.
    \end{equation*}
    So rearrange this sum,
    \begin{equation*}
        \psi(x)-\vartheta(x) = \sum_{m\geq 2} \left(\sum_{\substack{\pprime,\\ p\leq x^{1/m}}} \log{p}\right).
    \end{equation*}
    So by the definition of $\vartheta$,
    \begin{equation*}
        \psi(x)-\vartheta(x) = \sum_{m\geq 2} \vartheta(x^{1/m}).
    \end{equation*}
    However the sum on the right hand side is finite, since for $s < 2$, $\vartheta(s) = 0$. Hence, the sum
    may be written as
    \begin{equation*}
        \psi(x) - \vartheta(x) \leq \sum_{m=2}^{\lceil \log_2{x}\rceil} \vartheta(x^{1/m}).
    \end{equation*}
    Since for $m > \lceil \log_2{x} \rceil$, $x^{1/m} < 2$. Hence, since $\vartheta(s) \leq 2s$, we conclude
    \begin{equation*}
        \psi(x) - \vartheta(x) \leq 2\sum_{m=2}^{\lceil \log_2{x} \rceil} x^{\frac{1}{m}}.
    \end{equation*}
    Which we rewrite as 
    \begin{equation*}
        \psi(x) - \vartheta(x) \leq 2x^{\frac{1}{2}} + 2\sum_{m=3}^{\lceil \log_2{x} \rceil} x^{\frac{1}{m}}.
    \end{equation*}
    The largest term in the sum on the right hand side is $x^{\frac{1}{3}}$. So we have the bound,
    \begin{equation*}
        \psi(x) - \vartheta(x) \leq 2x^{\frac{1}{2}}+2x^{\frac{1}{3}}(\lceil \log_2{x} \rceil - 2) \leq 2x^{\frac{1}{2}} + 2x^{\frac{1}{3}} \log_2{x}.
    \end{equation*}
    However, we know from lemma \ref{logInequality} with $\alpha = 1/6$ that
    \begin{equation*}
        \log{x} \leq \frac{6x^{\frac{1}{6}}}{e}.
    \end{equation*}
    Hence,
    \begin{equation*}
        \psi(x) - \vartheta(x) \leq 2x^{\frac{1}{2}}+\frac{12}{e\log{2}}x^\frac{1}{2}  = \frac{2e\log{2}+12}{e\log{2}}x^\frac{1}{2}.
    \end{equation*}
    The coefficient on the right numerically evaluates as roughly $8.37$. So we have the bound,
    \begin{equation*}
        \psi(x) - \vartheta(x) \leq 9x^\frac{1}{2}.
    \end{equation*}
\end{proof}

\begin{theorem}
    We have the logical equivalence,
    \begin{equation*}
        \lim_{x\rightarrow\infty} \frac{\psi(x)}{x} = 1 \Longleftrightarrow \lim_{x\rightarrow\infty} \frac{\vartheta(x)}{x} = 1.
    \end{equation*}
\end{theorem}
\begin{proof}
    From theorem \ref{psiThetaBound}, 
    \begin{equation*}
        0 \leq \left| \left(\frac{\psi(x)}{x}-1\right)-\left(\frac{\vartheta(x)}{x}-1\right)\right| \leq 9x^{-\frac{1}{2}}.
    \end{equation*}
    By the reverse triangle inequality,
    \begin{equation*}
        0 \leq \left| \left|\frac{\psi(x)}{x}-1\right|-\left|\frac{\vartheta(x)}{x}-1\right|\right| \leq 9x^{-\frac{1}{2}}.
    \end{equation*}
    So if
    \begin{equation*}
        \lim_{x\rightarrow\infty} \frac{\psi(x)}{x} = 1.
    \end{equation*}
    Then for any $\varepsilon > 0$, we can find an $N > 0$ such that
    \begin{equation*}
        \left|\frac{\psi(x)}{x}-1\right| < \varepsilon
    \end{equation*}
    when $x > N$. Choose $M > N$ such that $9M^{-\frac{1}{2}} < \varepsilon$. Let $x > M$, then
    \begin{equation*}
        0 \leq \left| \frac{\vartheta(x)}{x}-1\right| \leq 2\varepsilon.
    \end{equation*}
    Since $\varepsilon$ is aribitrary, this implies that
    \begin{equation*}
        \lim_{x\rightarrow\infty} \frac{\vartheta(x)}{x} = 1.
    \end{equation*}
    Similarly, if the above limit holds, then
    \begin{equation*}
        \lim_{x\rightarrow\infty} \frac{\psi(x)}{x} = 1.
    \end{equation*}
\end{proof}


\section*{Question 2}
    \begin{theorem}
    \label{q2a}
        \begin{equation*}
            \lim_{x\rightarrow\infty} \frac{\log{\pi(x)}}{\log{x}} = 1.
        \end{equation*}
    \end{theorem}
    \begin{proof}
        Since
        \begin{equation*}
            \lim_{x\rightarrow\infty} \frac{\pi(x)\log{x}}{x} = 1, 
        \end{equation*}
        take a logarithm of both sides,
        \begin{equation*}
            \lim_{x\rightarrow\infty} \log\left(\frac{\pi(x)\log{x}}{x}\right) = 0
        \end{equation*}
        so that
        \begin{equation*}
            \lim_{x\rightarrow\infty} [\log{\pi(x)}+\log{\log{x}}-\log{x}] = 0.
        \end{equation*}
        Hence,
        \begin{equation*}
            \lim_{x\rightarrow\infty} \log{x}\left[\frac{\log{\pi(x)}}{\log{x}}+\frac{\log{\log{x}}}{\log{x}}-1\right] = 0.
        \end{equation*}
        However since $\log{x}$ increases without bound as $x\rightarrow\infty$, we must therefore have
        \begin{equation*}
            \lim_{x\rightarrow\infty} \left[\frac{\log{\pi(x)}}{\log{x}}+\frac{\log{\log{x}}}{\log{x}}-1\right] = 0.
        \end{equation*}
        By L'H\^opital's rule, $\lim_{x\rightarrow\infty} \frac{\log{\log{x}}}{\log{x}} = 0$. 
        Hence,
        \begin{equation*}
            \lim_{x\rightarrow\infty} \left[\frac{\log{\pi(x)}}{\log{x}}-1\right] = 0.
        \end{equation*}
        So the result follows.
    \end{proof}
    \begin{lemma}
    \label{q2c}
        Hence,
        \begin{equation*}
            \lim_{x\rightarrow\infty} \frac{\pi(x)\log{\pi(x)}}{x} = 1.
        \end{equation*}
    \end{lemma}
    \begin{proof}
        This follows from theorem \ref{q2a}. Since
        \begin{equation*}
            \lim_{x\rightarrow\infty} \frac{\pi(x)\log{x}}{x} = 1
        \end{equation*}
        and
        \begin{equation*}
            \lim_{x\rightarrow\infty} \frac{\log{\pi(x)}}{\log{x}} = 1
        \end{equation*}
        multiply these limits to obtain,
        \begin{equation*}
            \lim_{x\rightarrow\infty} \frac{\pi(x)\log{\pi(x)}}{x} = 1.
        \end{equation*}
    \end{proof}
    \begin{theorem}
        If $p_n$ denotes the $n$th prime, then
        \begin{equation*}   
            \lim_{n\rightarrow\infty} \frac{n\log{n}}{p_n} = 1.
        \end{equation*}
    \end{theorem}
    \begin{proof}
        From lemma \ref{q2c} 
        \begin{equation*}
            \lim_{x\rightarrow\infty} \frac{\pi(x)\log{\pi(x)}}{x} = 1.
        \end{equation*}
        Since $\{p_n\}_{n=1}^\infty$ is an increasing sequence, this implies
        \begin{equation*}
            \lim_{n\rightarrow\infty} \frac{\pi(p_n)\log{\pi(p_n)}}{p_n} = 1.
        \end{equation*}
        However, because $\pi(p_n) = n$, the result follows.
    \end{proof}
\section*{Question 3}
    \begin{remark}
        There are ``roughly" $\pi(n)$ primes between $n^2$ and $(n+1)^2$.
    \end{remark}
    This may or may not be true. But we can give a heuristic argument for the size of $\pi((n+1)^2)-\pi(n^2)$.
    We can suppose by the prime number theorem that there should be an approximate equivalence,
    \begin{equation*}
        \pi((n+1)^2)-\pi(n^2) \approx \frac{(n+1)^2}{\log{[(n+1)^2]}}-\frac{n^2}{\log{(n^2)}}
    \end{equation*}
    We may now simplify the right hand side as
    \begin{equation*}
        \frac{(n+1)^2\log{n}-n^2\log(n+1)}{2\log{n}\log(n+1)}
    \end{equation*}
    We wish to show that this is asymptotically equivalent to $\pi(n)$. Or, by the prime
    number theorem, we may show that this is asymptotically equivalent to $n/\log{n}$.    
    
    We must show that
    \begin{equation*}
        \frac{(n+1)^2\log{n}-n^2\log(n+1)}{2\log{n}\log(n+1)} \sim \frac{n}{\log{n}}
    \end{equation*}
    
    That is, it is necessary to prove that
    \begin{equation*}
        \lim_{n\rightarrow\infty} \frac{(n+1)^2\log{n}-n^2\log(n+1)}{2n\log(n+1)} = 1
    \end{equation*}
    
    So consider the numerator:
    \begin{equation*}
        (n+1)^2\log{n}-n^2\log(n+1).
    \end{equation*}
    By a Taylor expansion, we know that $\log(n+1) = \log(n) + \frac{1}{n}+\mathcal{O}\left(\frac{1}{n^2}\right)$. Hence,
    \begin{equation*}
        (n+1)^2\log{n}-n^2\log(n+1) = 2n\log{n}+\log{n}-n+\mathcal{O}(1)
    \end{equation*}
    Thus,
    \begin{align*}
        \lim_{n\rightarrow\infty} \frac{(n+1)^2\log{n}-n^2\log(n+1)}{2n\log(n+1)} &= \lim_{n\rightarrow\infty} \frac{2n\log{n}+\log{n}-n}{2n\log{(n+1)}}\\
        &= \lim_{n\rightarrow\infty} \frac{\log{n}}{\log{(n+1)}}+\lim_{n\rightarrow\infty} \frac{\log{n}-n}{2n\log{(n+1)}}\\
        &= 1+0\\
        &= 1
    \end{align*}
    as required.
\end{document}