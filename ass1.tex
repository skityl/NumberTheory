\documentclass[10pt]{article}
\usepackage{amsmath,amssymb,graphicx,color}
\title{Number theory Assignment 1}
\author{Edward McDonald}
\date{}

\newtheorem{theorem}{Theorem}
\newtheorem{lemma}[theorem]{Lemma}
\newtheorem{proposition}[theorem]{Proposition}
\newtheorem{corollary}[theorem]{Corollary}
\newenvironment{proof}[1][Proof]{\begin{trivlist}
\item[\hskip \labelsep {\bfseries #1}]}{\end{trivlist}}
\newenvironment{definition}[1][Definition]{\begin{trivlist}
\item[\hskip \labelsep {\bfseries #1}]}{\end{trivlist}}
\newenvironment{example}[1][Example]{\begin{trivlist}
\item[\hskip \labelsep {\bfseries #1}]}{\end{trivlist}}
\newenvironment{remark}[1][Remark]{\begin{trivlist}
\item[\hskip \labelsep {\bfseries #1}]}{\end{trivlist}}

\newcommand{\qed}{\nobreak \ifvmode \relax \else
	      \ifdim\lastskip<1.5em \hskip-\lastskip
		        \hskip1.5em plus0em minus0.5em \fi \nobreak
				      \vrule height0.75em width0.5em depth0.25em\fi}


\topmargin=-20mm
\textheight=250mm
\oddsidemargin=-4mm
\textwidth=166mm




\parskip=5pt
\parindent=0pt

\setlength{\parindent}{0pt}
\usepackage{fullpage}
\usepackage{enumerate}
\newcommand{\im}{\operatorname{im}}
\newcommand{\isom}{\cong}
\newcommand{\modulo}[1]{\;\operatorname{mod} #1}
\newcommand{\Char}{\operatorname{char}}
\newcommand{\tr}{\operatorname{tr}}
\newcommand{\dist}{\operatorname{dist}_{L^\infty}}

\newcommand{\legendre}[2]{\left(\frac{#1}{#2}\right)}

\begin{document}
\maketitle{}
    \section*{Question 1}
    For this question, let $n$ be a positive integer such that $\mathbb{U}_n$ has primitive roots 
        (That is, $n = 1,2,4$, a power of an odd prime or double a power of an odd prime).
    We also let $a$ be an integer coprime to $n$.
    \subsection*{Part a}
    \begin{theorem}

        
        $a \in \mathbb{U}_n$ is a $k$th power in $\mathbb{U}_n$ that is, there is
        a number $x \in \mathbb{U}_n$ with $x^k \equiv a\modulo{n}$ if and only if
        \begin{equation*}
            a^{\frac{\varphi(n)}{d}} \equiv 1\modulo{n}
        \end{equation*}
        where $\varphi$ is Euler's totient function and $d := (\varphi(n),k) := \gcd(\varphi(n),k)$.
                
    \end{theorem}
    \begin{proof}
        
        First suppose that $a$ is a $k$th power modulo $n$. So choose $x$
        such that
        \begin{equation*}
            a \equiv x^k\modulo{n}.
        \end{equation*}
        Note that $x$ must be coprime to $n$ due our assumption
        that $a$ is coprime to $n$.
        Then simply raise $a$ to the power $\varphi(n)/d$. So we have
        \begin{equation*}
            a^{\frac{\varphi(n)}{d}}\equiv x^{\frac{\varphi(n)k}{d}} \modulo{n}.
        \end{equation*}
        However the right hand side is $x^{\varphi(n)}$ raised to the power
        $k/d$. $k/d$ is a whole number since $d|k$. By Euler's theorem,
        $x^{\varphi(n)}\equiv 1 \modulo{n}$. Hence,
        \begin{equation*}
            a^{\frac{\varphi(n)}{d}} \equiv 1\modulo{n}.
        \end{equation*} 
        
        
        
        Conversely, suppose that
        \begin{equation*}
            a^{\frac{\varphi(n)}{d}}\equiv 1\modulo{n}
        \end{equation*}
        and we wish to find $x$ such that $a\equiv x^k\modulo{n}$. 
        
        By assumption, $\mathbb{U}_n$ has a primitive element (a generator). Let
        $\alpha$ be such a primitive element, and choose $r$ such that
        \begin{equation*}
            \alpha^r\equiv a\modulo{n}.
        \end{equation*}
        Raise both sides to the power $\varphi(n)/d$, to find
        \begin{equation*}
            \alpha^{\frac{r\varphi(n)}{d}}\equiv a^\frac{\varphi(n)}{d}\equiv 1\modulo{n}.
        \end{equation*}
        Since $\alpha$ is a primitive element, it must have minimal order $\varphi(n)$. Hence,
        we must have
        \begin{equation*}
            \varphi(n)|\frac{r\varphi(n)}{d}
        \end{equation*} 
        So choose $t\in\mathbb{Z}$ such that
        \begin{equation*}
            t\varphi(n) = \frac{r\varphi(n)}{d}.
        \end{equation*}
        
        Hence $r = td$. Recall that $d = (\varphi(n),k)$. So by Bezout's lemma, there
        are integers $p$ and $q$ such that $d = p\varphi(n)+qk$. 
        
        Thus,
        \begin{align*}
            a&\equiv\alpha^r\\
            &\equiv \alpha^{td}\\
            &\equiv \alpha^{tp\varphi(n)+tqk}\\
            &\equiv \alpha^{tp\varphi(n)}\alpha^{tqk}\\
            &\equiv \alpha^{tqk}\modulo{n}.
        \end{align*}
        So put $x = \alpha^{tq}$ and then $a\equiv x^k\modulo{n}$. $\Box$
       



        %First suppose that $a$ is a $k$th power modulo $n$, then since $a \equiv x^k\modulo{n}$ for some $x$, then
        %simply raise $a$ to the power $\varphi(n)/(k,\varphi(n))$. Hence,
        %\begin{equation*}
        %    a^{\frac{\varphi(n)}{(\varphi(n),k)}}\equiv x^{\frac{k\varphi(n)}{(\varphi(n),k)}}\modulo{n}.
        %\end{equation*}
        %Note that since $(a,\varphi(n)) = 1$, we must also have $(x,\varphi(n)) = 1$. So we can
        %apply Euler's theorem, $x^{\varphi(n)} \equiv 1\modulo{n}$. Hence,
        %\begin{equation*}
        %    a^{\frac{\varphi(n)}{(\varphi(n),k)}}\equiv1\modulo{n}.
        %\end{equation*}
        
        %Conversely, suppose that 
        %\begin{equation*}
        %    a^{\frac{\varphi(n)}{(\varphi(n),k)}}\equiv1\modulo{n}.
        %\end{equation*}
        %$\mathbb{U}_n$ has by assumption a primitive element, let $\alpha$ be a primitive
        %element for $\mathbb{U}_n$. Then choose $r$ such that
        %\begin{equation*}
        %    a \equiv \alpha^{r}\modulo{n}.
        %\end{equation*}
                
        %Now raise $a$ to the power $\varphi(n)/(\varphi(n),k)$,
        %\begin{equation*}
        %    a^{\frac{\varphi(n)}{(\varphi(n),k)}}\equiv \alpha^{\frac{r\varphi(n)}{(k,\varphi(n))}} \modulo{n}.
        %\end{equation*}
        %Hence,
        %\begin{equation*}
        %    \alpha^{\frac{r\varphi(n)}{(k,\varphi(n))}} \equiv 1 \modulo{n}.
        %\end{equation*}
        %But since $\alpha$ is a generator for the cyclic group $\mathbb{U}_n$, the order
        %of $\mathbb{U}_n$ must divide $\frac{r\varphi(n)}{(k,\varphi(n))}$. That is, there is an
        %integer $t$ such that $r\varphi(n) = (k,\varphi(n))\varphi(n)t$ since the order
        %of $\mathbb{U}_n$ is $\varphi(n)$. So $r = t(k,\varphi(n))$. By Bez\^out's lemma, 
        %there are integers $p$ and $q$ such that 
        %\begin{equation*}
        %    r = t(pk+q\varphi(n)).
        %\end{equation*}
       
        %Therefore, $a \equiv \alpha^r \equiv \alpha^{tpk}\modulo{n}$ since $\alpha^{tq\varphi(n)}\equiv 1\modulo{n}$. 
        %Thus, $a$ is a $k$th power modulo $n$. $\Box$
        
        
    \end{proof}
    
    We now wish to find the \emph{number}
    of solutions to the equation $x^k \equiv a\modulo{n}$, provided
    that it has solutions. Note that any two solutions, $x$ and $y$
    are related by a $k$th root of unity modulo $n$: since if $x^k\equiv y^k \modulo{n}$, then
    $xy^{-1}$ is a $k$th root of unity. Hence, given a solution
    $x$ we can find all solutions as $x\zeta$, where $\zeta$
    is a $k$th root of unity.
    
    So to find the number of solutions to the equation, we simply
    need to find the number of $k$th roots of unity modulo $n$.
    
    \begin{lemma}
        There are $(\varphi(n),k)$ $k$th roots of unity in $\mathbb{U}_n$.
    \end{lemma}
    \begin{proof}
        Consider the group homomorphism $\psi:\mathbb{U}_n\rightarrow\mathbb{U}_n$
        given by $x\mapsto x^k$.         Le $\alpha$ be a primitive element for $\mathbb{U}_n$.
        The image of $\psi$ is then
        \begin{equation*}
            \{1,\alpha^k,\alpha^{2k},\alpha^{3k},\ldots\}
        \end{equation*}
        This is a subgroup of $\mathbb{U}_n$, let its size be $r$. 
        $r$ must divide the size of $\mathbb{U}_n$, $r|\varphi(n)$, and $r$ is the smallest
        positive integer such that $\alpha^{kr}\equiv 1\modulo{n}$. 
        
        So $kr$ is the smallest multiple of $k$ divisible
        by $\varphi(n)$. Hence $kr = \operatorname{lcm}(k,\varphi(n))$.
        So $r = \varphi(n)/(\varphi(n),k)$.
        
        %Since $r$ divides $\varphi{n}$, put $r = \varphi(n)/q$. 
        
        %Then $\frac{\varphi(n)k}{q}$ is a multiple of $\varphi(n)$, since
        %$\alpha^{\varphi(n)k/q} \equiv 1\modulo{n}$ and the order of $\alpha$
        %is $\varphi(n)$. So we can find $t$ such that
        %\begin{equation*}
        %    \frac{\varphi(b)k}{q} = t\varphi(n).
        %\end{equation*} 
        %Hence $q|k$, and since $q|\varphi(n)$, we must have $q|(k,\varphi(n))$. 
        
        %However, since $r$ was defined as the smallest positive number
        %such that $\alpha^{rk} \equiv 1\modulo{n}$ and $r = \varphi(n)/q$, $q$
        %must be the greatest number such that $\alpha^{\varphi(n)k/q} \equiv 1\modulo{n}$
        %with $q | \varphi(n)$. 
        
        %We have proved that any such number $q$ must divide $k$ and $\varphi(n)$. 
        %Hence since $q$ is maximal we must have $q = (k,\varphi(n))$.
        
        So the image of $\psi$ has size $r = \varphi(n)/(k,\varphi(n))$.
        
        Hence the kernel of $\psi$ has size $(k,\varphi(n))$ by the first isomorphism
        theorem.
        
        The kernel of $\psi$ is exactly the numbers $x$ in $\mathbb{U}_n$
        such that $x^k \equiv 1\modulo{n}$, so there are $(k,\varphi(n))$ 
        $k$th roots of unity in $\mathbb{U}_n$. $\Box$

    \end{proof}
    
    By the above argument, if the equation $x^k\equiv a\modulo{n}$
    has solutions then there are exactly $(k,\varphi(n))$ solutions.
    
    \subsection*{Part b}
    \begin{corollary}
        If $p$ is a prime of the form $6k-1$, then the equation $x^3 \equiv a\modulo{p}$ has a unique solution
        for every $a$.
    \end{corollary}
    \begin{proof}
        In the case when $(a,p)\neq 1$, then $a \equiv 0\modulo{p}$, and so the equation $x^3\equiv a\modulo{p}$ has a unique
        solution $x = 0$. The solution is unique because the ring $\mathbb{Z}_p$ is a domain and cannot have nonzero
        nilpotent elements.
        
        For $(a,p) = 1$, we can use the results in part a.
        
        Put $p = 6k-1$. Then $\varphi(p) = 6k-2 = 2(3k-1)$. So $\varphi(p)$
        cannot be a multiple of $3$, since $3k-1$ is one less than a multiple of
        $3$. Hence, $(3,\varphi(p)) = 1$. Now by Euler's theorem
        \begin{equation*}
            a^{\varphi(p)/(3,\varphi(p))} \equiv a^{\varphi(p)} \equiv 1\modulo{p}.
        \end{equation*}
        
        Therefore, the equation $x^3 \equiv a\modulo{p}$ has a solution,
        and the number of solutions is $(\varphi(p),3) = 1$.
        
        That is, for any $a$ the equation $x^3 \equiv a\modulo{p}$ has a
        unique solution. $\Box$
        
        
    \end{proof}
   
    \section*{Question 2}
    For question $2$, suppose that $q$ is an odd prime such that $p := 2q+1$
    is also prime. For example, $q = 11$ and $p = 23$. 
    \begin{lemma}
        There are $q-1$ primitive roots modulo $p$, $q$ quadratic
        residues and $q$ quadratic non residues.
    \end{lemma}
    \begin{proof}
        The set $\mathbb{U}_p$ has $\varphi(\varphi(p))$ primitive roots. 
        Since $p = 2q+1$ is prime, $\varphi(p) = 2q$. As $q$ is odd, 
        we can compute $\varphi(2q) = \varphi(2)\varphi(q) = q-1$.
        Hence, there are $\varphi(\varphi(p)) = q-1$ primitive
        roots for $\mathbb{U}_n$.
        
        By Theorem 1.2 in chapter $1$ of the course notes, exactly
        half of the numbers in $\mathbb{U}_p$ are quadratic residues.
        Hence, the number of quadratic residues is half of $2q+1-1$. So
        the number of quadratic residues is $q$, and there must 
        also be $q$ quadratic non residues.
    \end{proof}
    \begin{lemma}
        A quadratic residue is never a primitive root.
    \end{lemma}
    \begin{proof}
        Suppose that $n$ is an integer, and $a$ is a quadratic
        residue in $\mathbb{U}_n$. So $a \equiv x^2\modulo{n}$
        for some $x\in\mathbb{U}_n$. If $a$ is a primitive root,
        then there is some $k$ such that $x \equiv a^k\modulo{n}$.
        Hence $a^k \equiv a^{2k}\modulo{n}$ and so $a^k \equiv 1\modulo{n}$. 
        Hence $x\equiv 1\modulo{n}$, so $a \equiv 1\modulo{n}$ and $a$
        cannot be a primitive root. $\Box$
    \end{proof}
    
    Since every primitive root is a quadratic non residue, 
    and here are $q-1$ primitive elements in $\mathbb{U}_p$ and $q$
    quadratic non residues: there must be exactly one
    number in $\mathbb{U}_n$ that is a quadratic non residue but
    not a primitive element.
    
    \begin{theorem}
        $2q \in \mathbb{U}_n$ is not a primitive element.
    \end{theorem}
    \begin{proof}
        Since $p = 2q+1$, $2q\equiv -1\modulo{p}$. Hence $2q$
        has minimal order $2$, and so cannot be a primitive element
        for $\mathbb{U}_n$. $\Box$
    \end{proof}
    
    \begin{theorem}
        $2q$ is a quadratic non residue modulo $p$.
    \end{theorem}
    \begin{proof}
        Since $2q\equiv -1\modulo{p}$, we simply
        need to show that $-1$ is not a quadratic residue
        modulo $p$. 
        
        By the corollary to Wilson's theorem in the course notes, 
        $-1$ is a quadratic residue modulo a prime $p$ precisely when $p\equiv 1\modulo{4}$.
        
        However, since $q$ is odd, we cannot have $2q+1\equiv 1\modulo{4}$. Hence
        $p$ is not equal to $1$ modulo $4$. So $-1$ is not a quadratic residue
        modulo $p$. $\Box$
    \end{proof}
    
    We have therefore shown that the unique quadratic non residue that
    is not a primitive element in $\mathbb{U}_p$
    is $2q$.
    
    \begin{corollary}
        The primitive roots of $\mathbb{U}_{23}$
        are $5,7,10,11,14,15,17,19,20,21$
    \end{corollary}
    \begin{proof}
        We simply need to find the quadratic non residues that
        are not $-1$, since $23$ is a prime of the form $2q+1$
        for an odd prime $q = 11$. We have already
        shown that there must be $q-1 = 10$ primitive roots.
        
        So we simply need to find a single primitive root, then
        all of the odd powers of that primitive root
        that are not $-1$ are the remaining primitive roots
        since even powers are quadratic residues.
        
        Consider $5$. We can show that $5$ is a primitive
        root by showing that it it not a quadratic residue.
        So we compute the Legendre symbol:
        \begin{equation*}
            \legendre{5}{23} = \legendre{23}{5}
        \end{equation*} 
        by quadratic reciprocity, since $5\equiv 1\modulo{4}$.
        The right hand side is $\legendre{3}{5}$.
        By quadratic reciprocity, this is $\legendre{5}{3} = \legendre{-1}{3}$.
        
        However by the corollary to Wilson's theorem, $\legendre{-1}{3} = -1$.
        Hence $5$ is not a quadratic residue modulo 23 and since $5\neq -1$,
        we can conlude that $5$ is a primitive element.
        
        Odd powers of $5$ are easily
        computed since $5^2\equiv 2\modulo{23}$. Hence the odd powers of $5$ are.
        \begin{align*}
            5^3 &\equiv 10\\
            5^5 &\equiv 20\\
            5^7 &\equiv 17\\
            5^9 &\equiv 11\\
            5^{11} &\equiv 22\equiv -1\\
            5^{13} &\equiv 21\\
            5^{15} &\equiv 19\\
            5^{17} &\equiv 15\\
            5^{19}&\equiv 7\\
            5^{21}&\equiv 14
        \end{align*}
        where the congruences are modulo $23$ .So excluding $5^{11}\equiv -1$, these are 
        the primitive elements of $\mathbb{U}_{23}$. $\Box$
        
        
    \end{proof}

    \section*{Question 3}
    \subsection*{Part a}
    \begin{theorem}
        Every number of the form $4n+2$ for some integer $n$
        is expressible as a sum of three squares, exactly
        $2$ of which are odd.
    \end{theorem}
    \begin{proof}
        By theorem 1.10 in the course notes, a number can be expressed
        as a sum of three squares unless it is of the form $4^\alpha(8k+7)$
        for some integers $\alpha, k$. Since $4n+2$ is not divisible
        by $4$, we need only show that $4n+2$ is not congruent to $7$
        modulo $8$.
        
        Suppose that $4n+2\equiv 7\modulo{8}$
        \begin{align*}
            4n+2&\equiv 7\modulo{8}\\
            \Rightarrow \;4n&\equiv5 \modulo{8}
        \end{align*}
        But then $5 = 4n-8k$ for some integer $k$. This is impossible
        because $5$ is odd. Hence, $4n+2$ is expressible
        as a sum of three squares.
        
        Suppose that $4n+2 = a^2 + b^2 + c^2$. We cannot have 
        all $a,b,c$ being even, because then $4|a^2+b^2+c^2$
        but $4n+2$ is not divisible by $4$. Similarly, if exactly
        one of $a,b,c$ is odd, then $a^2+b^2+c^2$ is odd, but $4n+2$
        is not odd. Similarly, if all of $a,b,c$ are odd
        then $4n+2$ is odd.
        
        Hence, the only possible case is that exactly two
        of $a,b,c$ are even. $\Box$
    \end{proof}
    \subsection*{Part b}
    \begin{theorem}
        Every odd positive integer can be expressed in the form $a^2+b^2+2c^2$
        for integers $a,b,c$.
    \end{theorem}
    \begin{proof}
        Suppose that $2n+1$ is any odd positive integer. Then by the preceding
        theorem there are integers $r,s,t$ such that
        \begin{equation*}
            4n+2 = (2r)^2+(2s+1)^2+(2t+1)^2.
        \end{equation*}
        Divide through by $2$ and expand, so that we have
        \begin{equation*}
            2n+1 = 2r^2+2s^2+2t^2+2s+2t+1.
        \end{equation*}
        
        Note the identity,
        \begin{equation*}
            (s+t)^2+(s-t)^2 = 2s^2+2t^2.
        \end{equation*}
        
        So we can express $2n+1$ as
        \begin{equation*}
            2n+1 = 2r^2+(s-t)^2+(s+t)^2+2s+2t+1.
        \end{equation*}
        Recognise that $(s+t)^2+2s+2t+1 = (s+t+1)^2$, so 
        \begin{equation*}
            2n+1 = 2r^2+(s-t)^2+(s+t+1)^2.
        \end{equation*}
        This gives the desired decomposition. $\Box$.
    \end{proof} 
    
    


\end{document}
