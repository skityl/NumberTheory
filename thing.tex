\documentclass[10pt]{article}
\usepackage{amsmath,amssymb,graphicx,color}
\title{That thing you keep on forgetting}
\author{}
\date{}

\newtheorem{theorem}{Theorem}
\newtheorem{lemma}[theorem]{Lemma}
\newtheorem{proposition}[theorem]{Proposition}
\newtheorem{corollary}[theorem]{Corollary}
\newenvironment{proof}[1][Proof]{\begin{trivlist}
\item[\hskip \labelsep {\bfseries #1}]}{\end{trivlist}}
\newenvironment{definition}[1][Definition]{\begin{trivlist}
\item[\hskip \labelsep {\bfseries #1}]}{\end{trivlist}}
\newenvironment{example}[1][Example]{\begin{trivlist}
\item[\hskip \labelsep {\bfseries #1}]}{\end{trivlist}}
\newenvironment{remark}[1][Remark]{\begin{trivlist}
\item[\hskip \labelsep {\bfseries #1}]}{\end{trivlist}}

\newcommand{\qed}{\nobreak \ifvmode \relax \else
	      \ifdim\lastskip<1.5em \hskip-\lastskip
		        \hskip1.5em plus0em minus0.5em \fi \nobreak
				      \vrule height0.75em width0.5em depth0.25em\fi}


\topmargin=-20mm
\textheight=250mm
\oddsidemargin=-4mm
\textwidth=166mm




\parskip=5pt
\parindent=0pt

\setlength{\parindent}{0pt}
\usepackage{fullpage}
\usepackage{enumerate}
\newcommand{\im}{\operatorname{im}}
\newcommand{\isom}{\cong}
\newcommand{\modulo}[1]{\;\operatorname{mod} #1}
\newcommand{\Char}{\operatorname{char}}
\newcommand{\tr}{\operatorname{tr}}
\newcommand{\dist}{\operatorname{dist}_{L^\infty}}

\newcommand{\legendre}[2]{\left(\frac{#1}{#2}\right)}

\begin{document}
\maketitle{}
    Let $n$ and $m$ be positive integers. Let
    $\varphi(x) = n\cdot x$ be a homomorphism
    on $\mathbb{Z}_m$. 
    \begin{theorem}
        $|\ker\varphi| = (n,m)$
    \end{theorem}
    \begin{proof}
        Let $k$ be the size of $\im\varphi$. Then 
        $k$ is the least positive integer such that $m|nk$. 
        
        So $nk$ is the least positive integer 
        that is a multiple of $m$ and $n$. That is, $k = \operatorname{lcm}(m,n)/n$.
        
        Hence $k = m/(n,m)$.
        
        
        %Since $\im\varphi$ is a subgroup of $\mathbb{Z}_m$, 
        %$k|m$. 
        
        %Put $qk = m$. Then $q$ is the maximal divisor of $m$
        %such that $m|\frac{nm}{q}$. 
        
        %Since $m|\frac{nm}{q}$, we must have $q|n$. 
        
        %Hence $q|(n,m)$ so $q$ cannot exceed $(n,m)$.
        %That is $q \leq (n,m)$.
        
        %$q$ is the largest divisor of $m$
        %such that $m|\frac{nm}{q}$. Since $m|\frac{nm}{(n,m)}$, we have $q \geq (n,m)$. 
        
        %Hence $q = (n,m)$.
                
        So $|\im\varphi| = m/(n,m)$.
        
        Thus, $|\ker\varphi| = (n,m)$. $\Box$
    \end{proof}
    
    
    \begin{example}
        The multiplicative group $\mathbb{F}_q^*$ is cyclic of
        order $q-1$. Hence it is isomorphic to $\mathbb{Z}_{q-1}$
        and the subgroup of $n$th roots of unity has order $(n,q-1)$.
    \end{example}
    \begin{example}
        Let $n$ be an integer such that $\mathbb{Z}_n^*$ is cyclic. Then
        $\mathbb{Z}_n$ has $(\varphi(n), k)$ $k$th roots of unity.
    \end{example}
    \begin{example}
        Suppose that we have an $n$ sided polygon and $k$ colours of paint.
        We wish to count the number of distinct ways of painting the sides
        of the polygon.
        
        The symmetry group of the polygon is $\mathbb{Z}_n$ and
        for each $x \in \mathbb{Z}_n$ the number of colourings
        invariant under rotation by $x$ is $k^{(n,x)}$. Why?
        
        Identify together edges of the polygon that are 
        congruent under the action of $x$. So identify together
        elements $a$ and $b$ of $\mathbb{Z}_n$ if $x|a-b$. 
        
        Hence the set of equivalence classes
        is the set of cosets of the subgroup $x\mathbb{Z}_n$. 
        
        The subgroup $x\mathbb{Z}_n$ has size $n/(x,n)$.
        
        Hence there are $(x,n)$ different colours 
        in a colouring invariant under $x$, so $k^{(x,n)}$ different
        colourings.
        
        Hence the total number of colourings, by the orbit counting
        formula (Burnside's lemma) is
        \begin{equation*}
            \frac{1}{n}\sum_{r = 0}^{n-1}k^{(n,r)}
        \end{equation*}
        
    \end{example}
    
    

    Now never forget it again.


\end{document}
