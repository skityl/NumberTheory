\documentclass{unswmaths}

\usepackage{unswshortcuts}

\begin{document}

\subject{Number Theory}
\author{Edward McDonald}
\title{Assignment 4}
\studentno{3375335}


\setlength\parindent{0pt}

\newcommand{\Unit}{\mathbb{U}}

\unswtitle{}

\section*{Question 1}
    \begin{lemma}
        Let $n > 0$. Suppose that $\{\chi_j\}_{j=1}^{\varphi(n)}$ are the characters
        of $\Unit_n$. Then for each $a,b\in \Unit_n$, we have the following orthogonality
        relation
        \begin{equation*}
            \sum_{j=1}^{\varphi(n)} \chi_j(a)\overline{\chi_j(b)} = \begin{cases}
                \varphi(n)\text{ if }a = b\\
                0\text{ overwise.}
            \end{cases}
        \end{equation*}
    \end{lemma}
    \begin{proof}
        Since for each $j$, $\overline{\chi_j(b)} = \chi_j(b^{-1})$, the sum
        \begin{equation*}
            \sum_{j=1}^{\varphi(n)} \chi_j(a)\overline{\chi_j(b)}
        \end{equation*}
        simplifies to
        \begin{equation*}
            \left(\sum_{j=1}^{\varphi(n)}\chi_j\right)(ab^{-1}).
        \end{equation*}
        Hence it is sufficient to prove that
        \begin{equation*}
            \left( \sum_{j=1}^{\varphi(n)} \chi_j\right)(a) = \begin{cases}
                \varphi(n)\text{ if }a = 1\\
                0\text{ otherwise.}
            \end{cases}
        \end{equation*}
        
        If $a = 1$, then for each $j$, $\chi_j(a) = 1$, so clearly
    \end{proof}
    

\end{document}
