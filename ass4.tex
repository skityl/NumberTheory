\documentclass{unswmaths}

\usepackage{unswshortcuts}

\begin{document}

\subject{Number Theory}
\author{Edward McDonald}
\title{Assignment 4}
\studentno{3375335}


\setlength\parindent{0pt}

\newcommand{\Unit}{\mathbb{U}}
\newcommand{\modulo}[1]{\;\operatorname{mod}\;#1}

\unswtitle{}

\section*{Question 1}
    \begin{lemma}
        \label{charOrth}
        Let $n > 0$. Suppose that $\{\chi_j\}_{j=1}^{\varphi(n)}$ are the characters
        of $\Unit_n$. Then for each $a,b\in \Unit_n$, we have the following orthogonality
        relation
        \begin{equation*}
            \sum_{j=1}^{\varphi(n)} \chi_j(a)\overline{\chi_j(b)} = \begin{cases}
                \varphi(n)\text{ if }a = b\\
                0\text{ overwise.}
            \end{cases}
        \end{equation*}
    \end{lemma}
    \begin{proof}
        Since each $\chi_i$ is a group homomorphism to the unit circle,
        we note the relation $\overline{\chi_j(b)} = \chi_j(b^{-1})$. 
        
        Let $k \in \{2,\ldots,\varphi(n)\}$. Then we simply permute
        the sum
        \begin{equation*}
            \sum_{j=1}^{\varphi(n)} \chi_j(a)\chi_j(b^{-1})
        \end{equation*}
        as
        \begin{equation*}
            \sum_{j=1}^{\varphi(n)} \chi_j(a)\chi_k(a)\chi_j(b^{-1})\chi_{k}(b^{-1}).
        \end{equation*}
        These sums are equal since the characters of $\Unit_n$ form a group, and pointwisely
        multiplying by $\chi_k$ simply permute the elements of the group. Then we have
        \begin{equation*}
            \sum_{j=1}^{\varphi(n)} \chi_j(ab^{-1}) = \chi_k(ab^{-1})\sum_{j=1}^{\varphi(n)} \chi_j(ab^{-1}).
        \end{equation*}
        If $a \neq b$, then since $k \neq 2$, we can find $k$ such that $\chi_k(ab^{-1}) \neq 1$. Hence
        \begin{equation*}
            \sum_{j=1}^{\varphi(n)} \chi_j(ab^{-1}) = 0
        \end{equation*}
        when $a\neq b$.
        
        If $a = b$, then we evaluate,
        \begin{align*}
            \sum_{j=1}^{\varphi(n)} \chi_j(ab^{-1}) &= \sum_{j=1}^{\varphi(n)} 1\\
            &= \varphi(n).
        \end{align*}
    \end{proof}
    \begin{lemma}
        Now suppose that $\chi_i:\Ntrl \rightarrow \Cplx$ are the Dirichlet characters on $\Unit_n$ for $i = 1,2,\ldots \varphi(n)$. Then we have for $a \in \Ntrl$ and $b\in \Unit_n$,
        \begin{equation*}
            \sum_{j=1}^{\varphi(n)} \chi_j(a)\overline{\chi_j(b)} = \begin{cases}
                \varphi(n)\text{ if }a \equiv b\modulo{n}\\
                0\text{ overwise.}
            \end{cases}
        \end{equation*}
    \end{lemma}
    \begin{proof}
        This follows from lemma \ref{charOrth} for $n \equiv a \modulo{n}$, 
    \end{proof}
    
    
    \begin{lemma}
        Let $\chi$ be a Dirichlet character, then
        
    \end{lemma}
    
    \section*{Question 2}
    \begin{proposition}
        Suppose $\chi$ is a Dirichlet character. Then
        \begin{equation*}
            \frac{1}{L(s,\chi)} = \sum_{n=1}^\infty \frac{\chi(n)\mu(n)}{n^s}
        \end{equation*}
    \end{proposition}
    \begin{proof}
        We can verify this by simply computing 
        \begin{equation*}
            L(s,\chi)\sum_{n=1}^\infty \frac{\chi(n)\mu(n)}{n^s}
        \end{equation*}
        By definition,
        \begin{align*}
            L(s,\chi)\sum_{n=1}^\infty \frac{\chi(n)\mu(n)}{n^s} &= \sum_{n=1}^\infty \frac{\chi(n)}{n^s}\sum_{n=1}^\infty \frac{\chi(n)\mu(n)}{n^s}\\
            &= \sum_{n=1}^\infty \frac{1}{n^s}\sum_{d|n} \chi\left(\frac{n}{d}\right)\chi(d)\mu(d)\\
            &= \sum_{n=1}^\infty \frac{1}{n^s}\sum_{d|n} \mu(d).
        \end{align*}
        Where the last line follows since characters are completely multiplicative. Now we use the identity
        \begin{equation*}
            \sum_{d|n}\mu(d) = I(n).
        \end{equation*}
        So,
        \begin{equation*}
            L(s,\chi)\sum_{n=1}^\infty \frac{\chi(n)\mu(n)}{n^s} = \sum_{n=1}^\infty \frac{I(n)}{n^s} = 1.
        \end{equation*}
        This completes the proof.
    \end{proof}
    
    \section*{Question 3}
    We consider the unique nonprincipal  dirichlet characters $\chi_4$ and $\chi_6$ modulo $4$ and $6$ respectively. That is, we have
    \begin{equation*}
        \chi_i(n) = \begin{cases}
            1\text{ if }n \equiv 1 \modulo{i}\\
            -1\text{ if }n \equiv -1 \modulo{i}\\
            0\text{ otherwise.}
        \end{cases}
    \end{equation*}
    For $i = 4,6$.
    \begin{theorem}
        $L(1,\chi_4) = \frac{\pi}{4}$
    \end{theorem}
    \begin{proof}
        We have the identity,
        \begin{equation*}
            
        \end{equation*}
    \end{proof}

\end{document}
