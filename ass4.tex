\documentclass{unswmaths}

\usepackage{unswshortcuts}

\begin{document}

\subject{Number Theory}
\author{Edward McDonald}
\title{Assignment 4}
\studentno{3375335}


\setlength\parindent{0pt}

\newcommand{\Unit}{\mathbb{U}}
\newcommand{\modulo}[1]{\;\operatorname{mod}\;#1}
\newcommand{\pprime}{{p\text{ prime}}}

\unswtitle{}

    
    \begin{definition}
        We have the following characters of $\Unit_5$:
        \\
        \\        
        \begin{tabular}{| l | cl  cr | r |}
        \hline
        $\Unit_5$ & $1$ & $2$ & $3$ & $4$\\
        \hline
        $\chi_1$  & $1$ & $1$ & $1$ & $1$\\
        $\chi_2$  & $1$ & $-1$ & $-1$ & $1$\\
        $\chi_3$  & $1$ & $i$ & $-i$ & $-1$\\
        $\chi_4$  & $1$ & $-i$ & $i$ & $-1$\\
        \hline
        \end{tabular}
    \end{definition}
    

\section*{Question 1}



    \begin{lemma}
        \label{charOrth}
        Let $n > 0$. Suppose that $\{\chi_j\}_{j=1}^{\varphi(n)}$ are the characters
        of $\Unit_n$. Then for each $a,b\in \Unit_n$, we have the following orthogonality
        relation
        \begin{equation*}
            \sum_{j=1}^{\varphi(n)} \chi_j(a)\overline{\chi_j(b)} = \begin{cases}
                \varphi(n)\text{ if }a = b\\
                0\text{ otherwise.}
            \end{cases}
        \end{equation*}
    \end{lemma}
    \begin{proof}
        Since each $\chi_i$ is a group homomorphism to the unit circle,
        we note the relation $\overline{\chi_j(b)} = \chi_j(b^{-1})$. 
        
        Let $k \in \{2,\ldots,\varphi(n)\}$. Then we simply permute
        the sum
        \begin{equation*}
            \sum_{j=1}^{\varphi(n)} \chi_j(a)\chi_j(b^{-1})
        \end{equation*}
        as
        \begin{equation*}
            \sum_{j=1}^{\varphi(n)} \chi_j(a)\chi_k(a)\chi_j(b^{-1})\chi_{k}(b^{-1}).
        \end{equation*}
        These sums are equal since the characters of $\Unit_n$ form a group, and pointwisely
        multiplying by $\chi_k$ simply permute the elements of the group. Then we have
        \begin{equation*}
            \sum_{j=1}^{\varphi(n)} \chi_j(ab^{-1}) = \chi_k(ab^{-1})\sum_{j=1}^{\varphi(n)} \chi_j(ab^{-1}).
        \end{equation*}
        If $a \neq b$, then since $k \neq 2$, we can find $k$ such that $\chi_k(ab^{-1}) \neq 1$. Hence
        \begin{equation*}
            \sum_{j=1}^{\varphi(n)} \chi_j(ab^{-1}) = 0
        \end{equation*}
        when $a\neq b$.
        
        If $a = b$, then we evaluate,
        \begin{align*}
            \sum_{j=1}^{\varphi(n)} \chi_j(ab^{-1}) &= \sum_{j=1}^{\varphi(n)} 1\\
            &= \varphi(n).
        \end{align*}
    \end{proof}
    \begin{lemma}
    \label{charOrth2}
        Now suppose that $\chi_i:\Ntrl \rightarrow \Cplx$ are the Dirichlet characters on $\Unit_n$ for $i = 1,2,\ldots \varphi(n)$. Then we have for $a \in \Ntrl$ and $b\in \Unit_n$,
        \begin{equation*}
            \sum_{j=1}^{\varphi(n)} \chi_j(a)\overline{\chi_j(b)} = \begin{cases}
                \varphi(n)\text{ if }a \equiv b\modulo{n}\\
                0\text{ otherwise.}
            \end{cases}
        \end{equation*}
    \end{lemma}
    \begin{proof}
        This follows from lemma \ref{charOrth}...
    \end{proof}
    

    \begin{lemma}
        For $\Re(s) > 1$, 
        \begin{equation*}
            \frac{1}{6} < L(s,\chi_2) < 1.
        \end{equation*}
    \end{lemma}
    \begin{proof}
        By definition,
        \begin{equation}
        \label{Lchi2}
            L(s,\chi_2) = 1-\frac{1}{2^s}-\frac{1}{3^s}+\frac{1}{4^s}+\frac{1}{6^s}-\frac{1}{7^s}-\frac{1}{8^s}+\frac{1}{9^s}+\frac{1}{11^s}+\cdots
        \end{equation}
        So from the second term onwards, we may bracket this sum in groups of four,
        \begin{equation*}
            L(s,\chi_2) = 1-\left(\frac{1}{2^s}+\frac{1}{3^s}-\frac{1}{4^s}-\frac{1}{6^s}\right)-\left(\frac{1}{7^s}+\frac{1}{8^s}-\frac{1}{9^s}-\frac{1}{11^s}\right)-\cdots
        \end{equation*}
        where each term in brackets is of the form,
        \begin{equation*}
            \frac{1}{k}+\frac{1}{(k+1)^s}-\frac{1}{(k+2)^s}-\frac{1}{(k+4)^s} > 0.
        \end{equation*}
        So we are subtracting a positive number from $1$. Hence, $L(s,\chi_2) > 1$. 
        
        To prove the second inequality, bracket the sum in \ref{Lchi2} as
        \begin{equation*}
            L(s,\chi_2) = \left(1-\frac{1}{2^s}-\frac{1}{3^s}\right)+\left(\frac{1}{4^s}+\frac{1}{6^s}-\frac{1}{7^s}-\frac{1}{8^s}\right)+\left(\frac{1}{9^s}+\frac{1}{11^s}+\cdots\right)
        \end{equation*}
        So that we are adding terms of the form $\frac{1}{k^s}+\frac{1}{(k^2)^s}-\frac{1}{(k+3)^s}-\frac{1}{(k+4)^s} > 0$ to $1-\frac{1}{2^s}-\frac{1}{3^s} > 1/6$. Hence $L(s,\chi_2) > 1/6$.
    \end{proof}
    \begin{lemma}
        For $\Re(s) > 1$,
        \begin{equation*}
            \frac{3}{4} < \Re(L(s,\chi_3)) = \Re(L(s,\chi_4)) < 1.
        \end{equation*}
    \end{lemma}
    \begin{proof}
        By definition,
        \begin{equation*}
            \Re(L(s,\chi_3)) = \Re(L(s,\chi_4)) = 1-\frac{1}{4^s}+\frac{1}{6^s}-\frac{1}{9^s}+\frac{1}{11^s}-\cdots
        \end{equation*}
        Bracket this sum in groups of two from the second term onwards,
        \begin{equation*}
            \Re(L(s,\chi_3)) = \Re(L(s,\chi_4)) = 1-(\frac{1}{4^s}-\frac{1}{6^s})-(\frac{1}{9^s}-\frac{1}{11^s})-\cdots
        \end{equation*}
        So we are subtracting positive terms of the form $\frac{1}{k^s}-\frac{1}{(k+2)^s}$ from $1$, hence $\Re(L(s,\chi_3)) < 1$
        and $\Re(L(s,\chi_4)) > 1$.
        
        Now bracket the sum from the third term onwards, 
        \begin{equation*}
            \Re(L(s,\chi_3)) = \Re(L(s,\chi_4)) = 1-\frac{1}{4^s}+(\frac{1}{6^s}-\frac{1}{9^s})+(\frac{1}{11^s}-\cdots
        \end{equation*}
        So each bracketed term is of the form $\frac{1}{k^s}-\frac{1}{(k+3)^s}$, which is positive. So $\Re(L(s,\chi_3)) = \Re(L(s,\chi_4)) > 1-\frac{1}{4^s} > \frac{3}{4}$
        for $\Re(s) > 1$.
    \end{proof}
    \begin{lemma}
        For $\frac{3}{2} > \Re(s) > 1$, 
        \begin{equation*}
            \frac{1}{2\sqrt{2}}-\frac{1}{3} < \Im(L(s,\chi_3)) = \Im(L(s,\chi_4)) < \frac{1}{2}
        \end{equation*}
    \end{lemma}
    \begin{proof}
        By definition,
        \begin{equation*}
            \Im(L(s,\chi_3)) = \Im(L(s,\chi_4)) = \frac{1}{2^s}-\frac{1}{3^s}+\frac{1}{7^s}-\frac{1}{8^s}+\cdots
        \end{equation*}
        If we bracket this sum in pairs from the second term onwards,
        \begin{equation*}
            \Im(L(s,\chi_3)) = \Im(L(s,\chi_4)) = \frac{1}{2^s}-\left(\frac{1}{3^s}-\frac{1}{7^s}\right)-\left(\frac{1}{8^s}-\frac{1}{12^s}\right)-\cdots
        \end{equation*}
        We are subtracting terms of the form $\frac{1}{k^s}-\frac{1}{(k+4)^s} > 0$ from $\frac{1}{2^s} < \frac{1}{2}$. Hence $\Im(L(s,\chi_3)) = \Im(L(s,\chi_4)) < \frac{1}{2}$.
        
        Now if we bracket in pairs from the third term onwards,
        \begin{equation*}
            \Im(L(s,\chi_3)) = \Im(L(s,\chi_4)) = \frac{1}{2^s}-\frac{1}{3^s}+\left(\frac{1}{7^s}-\frac{1}{8^s}\right)+\cdots
        \end{equation*}
        we are adding terms of the form $\frac{1}{k^s}-\frac{1}{(k+1)^s} > 0$ to $\frac{1}{2^s}-\frac{1}{3^s}$, hence for $\frac{3}{2} > \Re(s) > 1$, $\Im(L(s,\chi_4)) = \Im(L(s,\chi_3)) > \frac{1}{2\sqrt{2}}-\frac{1}{3}$
    \end{proof} 
    
    \begin{lemma}
    \label{euler}
        If $\chi$ is a dirichlet character, then for $\Re(s) > 1$,
        \begin{equation*}
            \log(L(s,\chi)) = \sum_\pprime \frac{\chi(p)}{p^s} + R(s)
        \end{equation*}
        where $|R(s)| < 1$.
    \end{lemma}
    \begin{proof}
        We use the Euler product identity,
        \begin{equation*}
            L(s,\chi) = \prod_{\pprime}\left(1-\frac{\chi(p)}{p^s}\right)^{-1}.
        \end{equation*}
        Since $\Re(s) > 1$, for any prime $p$, $0 < 1- \frac{1}{p^s} < 1$, so we may expand
        $\log(1-\frac{1}{p^s})$ as a power series,
        \begin{align*}
            \log(L(s,\chi)) &= -\sum_{\pprime} \log\left(1-\frac{\chi(p)}{p^s}\right)\\
            &= \sum_{\pprime} \sum_{n = 1}^\infty \frac{1}{n}\frac{\chi(p)^n}{p^{ns}}\\
            &= \sum_{\pprime} \frac{\chi(p)}{p^s} + \sum_{\pprime} \sum_{n=2}^\infty \frac{1}{n}\frac{\chi(p)^n}{p^{ns}}
        \end{align*}
        So define
        \begin{equation*}
            R(s) = \sum_{\pprime} \sum_{n=2}^\infty \frac{1}{n}\frac{\chi(p)^n}{p^{ns}}.
        \end{equation*}
        By the triangle inequality,
        \begin{equation*}
            |R(s)| \leq \sum_{\pprime} \sum_{n=2}^\infty \frac{1}{np^{ns}}
        \end{equation*}
        since for any $p$, $|\chi(p)| \leq 1$. Now we simply further bound for any $n > 1$,
        \begin{equation*}
            \frac{1}{np^{ns}} \leq \frac{1}{p^{ns}}
        \end{equation*}
        so that
        \begin{align*}
            |R(s)| &\leq \sum_{\pprime} \sum_{n=2}^\infty \frac{1}{p^{ns}}\\
            &= \sum_{\pprime} \frac{p^{-2s}}{1-p^{-s}}\\
            &= \sum_{\pprime} \frac{1}{p^s(p^s-1)}
        \end{align*}
        This is strictly bounded by,
        \begin{equation*}
            \sum_{n=2}^\infty \frac{1}{n^s(n^s-1)}.
        \end{equation*}
        Since $\Re(s) > 1$, we conclude that
        \begin{align*}
            |R(s)| &< \sum_{n=2}^\infty \frac{1}{n(n-1)}\\
            &= \sum_{n=2}^\infty \frac{1}{n-1}-\frac{1}{n}\\
            &= 1.
        \end{align*}
        Where the last equality follows since the sum telescopes.
    \end{proof}
    
    \begin{theorem}
        If $a \in \Unit_5$, then the sum
        \begin{equation*}
            \sum_{\pprime, p \equiv a \modulo{5}} \frac{1}{p}
        \end{equation*}
        diverges, and hence there are infinitely many primes congruent to $a$ modulo $5$.
    \end{theorem}
    \begin{proof}
        For the Dirichlet characters $\chi_i$ of $\Unit_5$ for $i = 1,2,3,4$, from lemma \ref{euler} we know that
        \begin{equation*}
        \label{eulerL}
            \log(L(s,\chi_i)) = \sum_{\pprime} \frac{\chi_i(p)}{p^s}+R_i(s)
        \end{equation*}
        with $|R_i(s)| < 1$. By lemma \ref{charOrth2}, we can extract primes congruent to $a$ modulo $5$
        by taking a linear combination of equations \ref{eulerL} as follows,
        \begin{align*}
            \sum_{i=1}^4 \log(L(s,\chi_i))\overline{\chi_i(a)} &= \sum_{\pprime} \frac{1}{p^s}\left(\sum_{i=1}^4 \chi_i(p)\overline{\chi_i(a)}\right) + \sum_{i=1}^4 R_i(s)\overline{\chi_i(a)}\\
            &= \sum_{\pprime, p\equiv a\modulo{p}} \frac{1}{p^s} + R(s).
        \end{align*}
        where $R(s) := \sum_{i=1}^4 R_i(s)\overline{\chi_i(a)}$, and by the triangle inequality $|R(s)| < 4$.
        
        So since $L(s,\chi_1) \rightarrow \infty$ as $s\rightarrow 1$, and the remaining terms are bounded, we
        conlude that the series
        \begin{equation*}
            \sum_{\pprime, p\equiv a\modulo{p}} \frac{1}{p^s}
        \end{equation*}
        becomes arbitrarily large as $s\rightarrow 1$. Hence, 
        the sum
        \begin{equation*}
            \sum_{\pprime, p\equiv a\modulo{p}} \frac{1}{p}
        \end{equation*}
        diverges.
    \end{proof} 
    
    
    \section*{Question 2}
    \begin{proposition}
        Suppose $\chi$ is a Dirichlet character. Then
        \begin{equation*}
            \frac{1}{L(s,\chi)} = \sum_{n=1}^\infty \frac{\chi(n)\mu(n)}{n^s}
        \end{equation*}
    \end{proposition}
    \begin{proof}
        We can verify this by simply computing 
        \begin{equation*}
            L(s,\chi)\sum_{n=1}^\infty \frac{\chi(n)\mu(n)}{n^s}
        \end{equation*}
        By definition,
        \begin{align*}
            L(s,\chi)\sum_{n=1}^\infty \frac{\chi(n)\mu(n)}{n^s} &= \sum_{n=1}^\infty \frac{\chi(n)}{n^s}\sum_{n=1}^\infty \frac{\chi(n)\mu(n)}{n^s}\\
            &= \sum_{n=1}^\infty \frac{1}{n^s}\sum_{d|n} \chi\left(\frac{n}{d}\right)\chi(d)\mu(d)\\
            &= \sum_{n=1}^\infty \frac{1}{n^s}\sum_{d|n} \mu(d).
        \end{align*}
        Where the last line follows since characters are completely multiplicative. Now we use the identity
        \begin{equation*}
            \sum_{d|n}\mu(d) = I(n).
        \end{equation*}
        So,
        \begin{equation*}
            L(s,\chi)\sum_{n=1}^\infty \frac{\chi(n)\mu(n)}{n^s} = \sum_{n=1}^\infty \frac{I(n)}{n^s} = 1.
        \end{equation*}
        This completes the proof.
    \end{proof}
    
    \section*{Question 3}
    We consider the unique nonprincipal  dirichlet characters $\chi_4$ and $\chi_6$ modulo $4$ and $6$ respectively. That is, we have
    \begin{equation*}
        \chi_i(n) = \begin{cases}
            1\text{ if }n \equiv 1 \modulo{i}\\
            -1\text{ if }n \equiv -1 \modulo{i}\\
            0\text{ otherwise.}
        \end{cases}
    \end{equation*}
    For $i = 4,6$.
    \begin{theorem}
        $L(1,\chi_4) = \frac{\pi}{4}$
    \end{theorem}
    \begin{proof}
        We have the identity,
        \begin{equation*}
            L(1,\chi(4)) = \int_0^1 \frac{\lambda(t)}{1-t^4}\;dt
        \end{equation*}
        where
        \begin{align*}
            \lambda(t) &= \sum_{n=1}^4 \chi_4(n)t^{n-1}\\
            &= 1-t^2.
        \end{align*}
        Hence,
        \begin{align*}
            L(1,\chi_4) &= \int_0^1 \frac{1-t^2}{1-t^4}\;dt\\
            &= \int_0^1 \frac{1}{1+t^2}\;dt\\
            &= \tan^{-1}(1)-\tan^{-1}(0)\\
            &= \frac{\pi}{4}.
        \end{align*}        
    \end{proof}
    
    \begin{theorem}
        $L(1,\chi_6) = \frac{\pi}{2\sqrt{3}}$
    \end{theorem}
    \begin{proof}
        Once again we use the identity,
        \begin{equation*}
            L(1,\chi_6) = \int_0^1 \frac{\lambda(t)}{1-t^6}\;dt
        \end{equation*}
        where now $\lambda(t)= \sum_{n=1}^6 \chi_6(n)t^{n-1} = 1-t^4$. So
        \begin{equation*}
            L(1,\chi_6) = \int_0^1 \frac{1+t^2}{1+t^2+t^4}\;dt.
        \end{equation*}
        We can evaluate this integral simply by noting that
        \begin{equation*}
            \frac{1+t^2}{1+t^2+t^4} = \frac{1}{2}\left(\frac{1}{1+t+t^2}+\frac{1}{1-t+t^2}\right).
        \end{equation*}
        Then we have
        \begin{align*}
            L(1,\chi_6) &= \frac{1}{2}\int_0^1 \frac{1}{1+t+t^2}\;dt+\frac{1}{2}\int_0^1 \frac{1}{1-t+t^2}\;dt\\
            &= \frac{1}{2}(\frac{2}{\sqrt{3}}\tan^{-1}(\frac{2}{\sqrt{3}}\frac{3}{2})-\frac{2}{\sqrt{3}}\tan^{-1}(\frac{2}{\sqrt{3}}\frac{1}{2}))+\frac{1}{2}(\frac{2}{\sqrt{3}}\tan^{-1}(\frac{2}{\sqrt{3}}\frac{1}{2})-\frac{2}{\sqrt{3}}\tan^{-1}(-\frac{2}{\sqrt{3}}\frac{1}{2}))\\
            &= \frac{\pi}{2\sqrt{3}}
        \end{align*}
    \end{proof}

\end{document}
