\documentclass{unswmaths}

\usepackage{unswshortcuts}

\begin{document}

\subject{Number Theory}
\author{Edward McDonald}
\title{Assignment 5}
\studentno{3375335}


\setlength\parindent{0pt}

\newcommand{\Unit}{\mathbb{U}}
\newcommand{\modulo}[1]{\;\operatorname{mod}\;#1}
\newcommand{\pprime}{{p\text{ prime}}}

\unswtitle{}


\section*{Question 1}
    \begin{lemma}
    \label{zetaBounds}
        For $s > 1$, 
        \begin{equation*}
            \frac{1}{s-1} \leq \zeta(s) \leq 1+\frac{1}{s-1}
        \end{equation*}
    \end{lemma}
    \begin{proof}
        We use the formula,
        \begin{equation*}
            \zeta(s) = \frac{s}{s-1}-s\int_{1}^\infty \frac{\{x\}}{x^{s+1}}\;dx.
        \end{equation*}
        Then since for $x \geq 1$, $\frac{\{x\}}{x^{s+1}} \geq 0$, and $s > 1$, we have
        \begin{equation*}
            \zeta(s) \leq \frac{s}{s-1} = 1 + \frac{1}{s-1}.
        \end{equation*}
        
        Now, using $\{x\} \leq 1$, we have
        \begin{equation*}
            s\int_1^\infty \frac{\{x\}}{x^{s+1}}\;dx \leq \int_{1}^\infty sx^{-s-1}\;dx = 1.
        \end{equation*}
        So,
        \begin{equation*}
            \zeta(s) \geq \frac{s}{s-1}-1 = \frac{1}{s-1}.
        \end{equation*}
    \end{proof}
    
    \begin{lemma}
        Hence,
        \begin{equation*}
            \lim_{s\rightarrow 1^+} (s-1)\zeta(s) = 1.
        \end{equation*}
    \end{lemma}
    \begin{proof}
        By lemma \ref{zetaBounds}, 
        \begin{equation*}
            1 \leq (s-1)\zeta(s) \leq (s-1)+1
        \end{equation*}
        for $s > 1$. Then by the pinching theorem,
        \begin{equation*}
            1 \leq \lim_{s\rightarrow 1^+} (s-1)\zeta(s) \leq 1+\lim_{s\rightarrow 1^+}(s-1) = 1.
        \end{equation*}
        Hence,
        \begin{equation*}
            \lim_{s\rightarrow 1^+} (s-1)\zeta(s) = 1.
        \end{equation*}      
    \end{proof}
    
    \begin{lemma}
    \label{zetaLimit}
        \begin{equation*}
            \lim_{s\rightarrow 1^+} \frac{\log\zeta(s)}{\log\frac{1}{s-1}} = 1.
        \end{equation*}
    \end{lemma}
    \begin{proof}
        By lemma \ref{zetaBounds}, 
        \begin{equation*}
            \frac{1}{s-1} \leq \zeta(s) \leq 1+\frac{1}{s-1} = \frac{s}{s-1}.
        \end{equation*}
        So take logarithms,
        \begin{equation*}
            \log\frac{1}{s-1} \leq \log\zeta(s) \leq \log(s) +\log\frac{1}{s-1}
        \end{equation*}
        Assume that $s < 2$, so that $\log\frac{1}{s-1} > 0$. Now divide by $\log\frac{1}{s-1}$,
        \begin{equation*}
            1 \leq \frac{\log\zeta(s)}{\log\frac{1}{s-1}} \leq 1 - \frac{\log(s)}{\log(s-1)}
        \end{equation*}
        Now as $s\rightarrow 1^+$, $\frac{\log(s)}{\log(s-1)}\rightarrow 0$.
        Hence by the pinching theorem,
        \begin{equation*}
            \lim_{s\rightarrow 1^+} \frac{\log\zeta(s)}{\log\frac{1}{s-1}} = 1.
        \end{equation*}
    \end{proof}
    
    \begin{theorem}
        The Dirichlet density of the set of all primes,
        \begin{equation*}
            \lim_{s\rightarrow 1^+} \frac{\sum_{p\;\mathrm{prime}} \frac{1}{p^s}}{\log\frac{1}{s-1}}
        \end{equation*}
        is $1$. 
    \end{theorem}
    \begin{proof}
        We may write 
        \begin{equation*}
            \log\zeta(s) = -\sum_{\pprime} \log\left(1-p^{-s}\right) = \sum_{\pprime}\frac{1}{p^s}+R(s)
        \end{equation*}
        where $|R(s)| < 1$, as covered in lectures.
        
        Hence,
        \begin{equation*}
            \lim_{s\rightarrow 1^+} \frac{\sum_{\pprime}\frac{1}{p^s}}{\log\frac{1}{s-1}} = \lim_{s\rightarrow 1^+} \frac{\log\zeta(s)-R(s)}{\log\frac{1}{s-1}}
        \end{equation*}
        So since $|R(s)| \leq 1$, we have
        \begin{equation*}
            \lim_{s\rightarrow 1^+} \frac{R(s)}{\log\frac{1}{s-1}} = 0.
        \end{equation*}
        Hence,
        \begin{equation*}
            \lim_{s\rightarrow 1^+} \frac{\sum_{\pprime}\frac{1}{p^s}}{\log\frac{1}{s-1}} = \lim_{s\rightarrow 1^+} \frac{\log\zeta(s)}{\log\frac{1}{s-1}} = 1
        \end{equation*}
        by lemma \ref{zetaLimit}. 
    \end{proof}
    
    \section*{Question 2}
    \begin{lemma}
    \label{q2a}
        If $p$ is prime, and $s > 2$, then
        \begin{equation*}
            \sum_{n\geq 0} \frac{\varphi(p^n)}{p^{ns}} = \frac{1-p^{-s}}{1-p^{1-s}}
        \end{equation*}
    \end{lemma}
    \begin{proof}
        The sum
        \begin{equation*}
            \sum_{n=0}^\infty \frac{\varphi(p^n)}{p^{ns}}
        \end{equation*}
        Can be written as a geometric series, since $\varphi(p^n) = p^n-p^{n-1}$ for $n > 0$,
        \begin{equation*}
            \sum_{n=0}^\infty \frac{\varphi(p^n)}{p^{ns}} = 1+\sum_{n=1}^\infty \frac{p^n-p^{n-1}}{p^{ns}}
        \end{equation*}
        So this is simply,
        \begin{equation*}
            \sum_{n\geq 0} \frac{\varphi(p^n)}{p^{ns}} = 1+\sum_{n=1}^\infty \frac{1}{p^{n(s-1)}} - \frac{1}{p}\sum_{n=1}^\infty \frac{1}{p^{n(s-1)}}
        \end{equation*}
        Hence,
        \begin{equation*}
            \sum_{n\geq 0} \frac{\varphi(p^n)}{p^{{ns}}} = 1+\frac{p^{1-s}}{1-p^{1-s}}-\frac{p^{-s}}{1-p^{1-s}} = \frac{1-p^{-s}}{1-p^{1-s}}.
        \end{equation*}
    \end{proof}
    \begin{theorem}
    So for $s > 2$, 
        \begin{equation*}
            \sum_{n = 1}^\infty \frac{\varphi(n)}{n^s}  = \frac{\zeta(s-1)}{\zeta(s)}
       \end{equation*}
    \end{theorem}
    \begin{proof}
        Since $\varphi$ is a mulltiplicative function, we have the Euler product formula,
        \begin{equation*}
            \sum_{n=1}^\infty \frac{\varphi(n)}{n^s} = \prod_{\pprime} \left(\sum_{n\geq 0}\frac{\varphi(p^n)}{p^{ns}}\right)
        \end{equation*}
        Hence, by lemma \ref{q2a},
        \begin{equation*}
            \sum_{n=1}^\infty \frac{\varphi(n)}{n^s} = \prod_{\pprime} \frac{1-p^{-s}}{1-p^{-(s-1}}.
        \end{equation*}
        But since,
        \begin{equation*}
            \zeta(s) = \prod_{\pprime} (1-p^{-s})^{-1},
        \end{equation*}
        this means
        \begin{equation*}
            \sum_{n=1}^\infty \frac{\varphi(n)}{n^s} = \frac{\zeta(s-1)}{\zeta(s)}.
        \end{equation*}
    \end{proof}
    
\section*{Question 3}
Define
\begin{equation*}
    M(x) = \sum_{n\leq x} \mu(n).
\end{equation*}
\begin{theorem} 
\label{zetaInt}
    For $\Re(s) > 1$,
    \begin{equation*}
        \frac{1}{\zeta(s)} = s\int_{1}^\infty M(x)x^{-s-1}\;dx.
    \end{equation*}
\end{theorem}
\begin{proof}
    Rewrite the formula,
    \begin{equation*}
        \frac{1}{\zeta(s)} = \sum_{n=1}^\infty \frac{\mu(n)}{n^s}
    \end{equation*}
    as
    \begin{equation*}
        \frac{1}{\zeta(s)} = \sum_{n=1}^\infty \int_{n}^\infty s\mu(n)x^{-s-1}\;dx = s\sum_{n=1}^\infty \int_{n}^\infty \mu(n)x^{-s-1}\;dx.
    \end{equation*}
    
    We can now interchange the integral and the sum, 
    \begin{equation*}
        \frac{1}{\zeta(s)} = s\int_{1}^\infty \sum_{n\leq x} \mu(n) x^{-s-1}\;dx.
    \end{equation*}
    as required.
\end{proof}
\begin{lemma}
    If $M(x) = \mathcal{O}(x^{\frac{1}{2}+\varepsilon})$ for all $\varepsilon > 0$, then
    the integral in theorem \ref{zetaInt} converges for all $s$ with $\Re(s) > 1/2+\varepsilon$.
\end{lemma}
\begin{proof}
    Since $M(x)x^{-s-1} > 0$ for all $x \in (1,\infty)$ it is sufficient to bound the
    function $x\mapsto M(x)x^{-s-1}$ from above.
    
    If $M(x) = \mathcal{O}(x^{\frac{1}{2}+\varepsilon})$ for any $\varepsilon > 0$,
    then there exist constants $a$ and $C$ depending on $\varepsilon$
    such that $M(x) \leq Cx^{\frac{1}{2}+\varepsilon}$ for all $x > a$. 
   
    Then $M(x)x^{-s-1} \leq Cx^{-s-1/2+\varepsilon}$ for $x > a$. So the integral
    \begin{equation*}
        \int_{1}^\infty M(x)x^{-s-1}\;dx = \int_{1}^a M(x)x^{-s-1}\;dx+\int_{a}^\infty M(x)x^{-s-1}\;dx
    \end{equation*}    
    is finite, since
    \begin{equation*}
        \int_{a}^\infty x^{-s-1/2+\varepsilon}\;dx
    \end{equation*}
    is finite when $s > 1/2+\varepsilon$.
\end{proof}

\begin{theorem}
    If $M(x) = \mathcal{O}(x^{\frac{1}{2}+\varepsilon})$ for any $\varepsilon > 0$, then 
    there are no zeroes of $\zeta(s)$ with $\Re(s) \neq 1/2$ with $\Re(s) \geq 0$.
\end{theorem}
\begin{proof}
    Since the function
    \begin{equation*}
        G(s) := s\int_{1}^\infty M(x)x^{-s-1}\;dx
    \end{equation*}
    agrees with $1/\zeta(s)$ for $\Re(s) > 1$ and is analytic and if $M(x) = \mathcal{O}(x^{\frac{1}{2}+\varepsilon})$,
    then this function is defined for $\Re(s) > 1/2+\varepsilon$ for any $\varepsilon > 0$ and so is an analytic continuation 
    of $\zeta(s)$ to the region $\{s \in \Cplx\;:\;\Re(s) > 1/2\;\}$.
    
    Hence, the analytic continuation of $\zeta(s)$ is nonzero in the region $\{s \in \Cplx\;:\;\Re(s) > 1/2\;\}$, and so by the reflection formula, is analytic in the region $\{s\in\Cplx\;:\;\Re(s) \geq 0\;\}$.
    
    So $\zeta(s)$ cannot have any zeros with nonnegative real part with real part not equal to $1/2$.
\end{proof}


\end{document}
